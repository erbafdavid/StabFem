%\documentclass[a4paper,12pt]{article}
\documentclass[showpacs,preprintnumbers,amsmath,amssymb,aps]{revtex4-1}
%\setlength{\textwidth}{17.2cm}
%\setlength{\textheight}{22.8cm}
%\setlength{\oddsidemargin}{-.3cm}
%\setlength{\evensidemargin}{-.3cm}
%\documentclass[10pt,twocolumn]{article}
%\documentclass[floatfix,twocolumn,aps,pra,superscriptaddress,showpacs]{revtex}
%\usepackage[utf8x]{inputenc}
\usepackage{caption}
%\usepackage[font=small,format=plain,labelfont=bf,up,textfont=normal,up,justification=justified,singlelinecheck=false]{caption}
\usepackage{default}
\usepackage{amssymb}
\usepackage{color}
\usepackage{amsmath}
%\usepackage[spanish]{babel}
\usepackage{mathrsfs}
\usepackage{graphicx}
\usepackage{subfig}
\usepackage{epsfig}
% Title Page




%\documentclass[aip,pof,amsmath,amssymb,preprint]{revtex4-1}

%\usepackage{graphicx}% Include figure files
%\usepackage{dcolumn}% Align table columns on decimal point
%\usepackage{bm}% bold math

%\usepackage{graphicx}
%\usepackage[T1]{fontenc}

\begin{document}

\title{{\bf Linear stability analysis of a disc: effect of porosity.}} 
\author{D. Lo Jacono}
\affiliation{
Institut de M\'ecanique des Fluides de Toulouse (IMFT), Universit\'e de Toulouse, CNRS, Toulouse, France} 
\author{D. Fabre}
\affiliation{
Institut de M\'ecanique des Fluides de Toulouse (IMFT), Universit\'e de Toulouse, CNRS, Toulouse, France} 
\date{\today}

% ----------------------------------------------------------------------------
\begin{abstract}
Axisymmetric wake of a disc placed normal to the incoming uniform flow. Focus from the trivial solution upto the onset of several unstable sutuations. Two main parameter is explored (Reynolds number, and porosity). Maybe we will explore some swirl. what about aspect-ratio. Understand basic mechanism of classical scenario with the effect of a porosiy. 
\end{abstract}   
% ---------------------------------------------------------------------------
\maketitle

\section{Introduction}


\section{Governing equations}\label{equations}
axisymmetric



\begin{figure}
\caption{Sketch of the geometry .\label{sketch}}
\end{figure}
Aspect-ratio of disc: 0.3 . thickness over diameter. 


\section{Numerical method}
\label{numerics}
FreeFem+
here modelling of the porosity.
\cite{thompson:96}
\section{Results}
\label{results}
\subsection{The solid case: $\alpha=1$}


\subsection{The porous case: $\alpha<1$}


\subsection{Direct numerical simulations for $\alpha=1$}
some DNS simulations?

\section{Conclusions}\label{conclusions}

\bibliography{lit}


\end{document}

